\documentclass[12pt,a4paper]{article}
\usepackage{listings,xcolor,setspace,hyperref,dirtree}
\usepackage{catchfile,pgffor,graphicx,float,caption}
\usepackage[margin=2.5cm]{geometry} % Set page margin to 2.5cm
\usepackage{etoolbox}
\usepackage{mathptmx} % Use Times New Roman font
\usepackage[style=numeric,backend=biber]{biblatex}

\addbibresource{paper/bibliography.bib}

\author{Florian M. Donnelly}
\title{The Development of a Programming Language}

% Specify the version
\newcommand{\ver}[1]{
    \expandafter\ifstrequal\expandafter{\jobname}{paper}
    {#1}{}
}

\definecolor{mygreen}{rgb}{0.3, 0.5, 0.2}
\definecolor{myteal}{rgb}{0.1, 0.3, 0.5}
\definecolor{myblue}{rgb}{0.2, 0.0, 0.6}
\definecolor{mypink}{rgb}{0.8, 0.3, 0.6}
\definecolor{myorange}{rgb}{0.4, 0.2, 0.0}

% Use fancier hyperrefs
\hypersetup{
    colorlinks=true,
    citecolor=mygreen,
    linkcolor=myteal,
    urlcolor=myblue,
}

% Define language for listings
\lstdefinelanguage{tslang}{
  keywords={std},
  morecomment=[l]{//},
  morecomment=[s]{/*}{*/},
  morestring=[b]',
  morestring=[b]",
  ndkeywords={},
  sensitive=true
}


% hyperref abbreviations
\makeatletter
\DeclareRobustCommand{\filename}[1]{%
    \begingroup
    \def\textendash{-}%
    \filename@parse{#1}%
    \edef\filename@base{\detokenize\expandafter{\filename@base}}%
    \texttt{\filename@base.\filename@ext}%
    \endgroup
}
\makeatother
\newcommand{\hr}[2]{\hyperref[#2]{#1}}
\newcommand{\hrc}[1]{\hyperref[#1]{\filename{#1}}}

% Read contents of files.txt into \files
\CatchFileDef{\files}{paper/files.txt}{\endlinechar=-1}
\CatchFileDef{\githash}{paper/githash.txt}{\endlinechar=-1}

% Include C code from file #1 from lines #2 to #3
\lstset{basicstyle=\scriptsize,keywordstyle=\color{myorange},stepnumber=5,
        frame=none,keepspaces=false,showstringspaces=false,
        rulecolor=\color{black},tabsize=2,
        stringstyle=\color{mygreen},commentstyle=\color{mypink},
        morekeywords={NULL},
        morecomment=[l][\color{magenta}]{\#},
        postbreak=\mbox{\textcolor{red}{$\hookrightarrow$}\space}}

\newcommand{\sourcecode}[4] {
    \pagelabel{#1}
    \lstinputlisting[language=#4,numbers=left,breaklines=true,
    frame=single,title=Contents of the file #1:,firstnumber=#2,firstline=#2,lastline=#3]{#1}
}

\newcommand{\code}[4] {
    \ver{

    \centerline{\begin{minipage}{\linewidth}
    \lstinputlisting[language=#4,numbers=left,breaklines=true,
        frame=single,title=\hr{#1}{#1},
        firstnumber=#2,firstline=#2,lastline=#3]{#1}
    \end{minipage}}

    }
}

% remove heading from lof
\makeatletter
\renewcommand\listoffigures{%
        \@starttoc{lof}%
}
\makeatother

% set caption width
\captionsetup{width=.7\linewidth}


% Cite text
\DeclareCiteCommand{\cte}
  {\usebibmacro{prenote}}
  {\usebibmacro{citeindex}%
   \usebibmacro{cite}}
  {\multicitedelim}
  {\usebibmacro{postnote}}

% Center text on newline
\newcommand{\expr}[1] {
    \begin{center}
        #1
    \end{center}
}

\newcommand{\paste}[1]{
    \begin{tabular}{c}
    \lstinputlisting[inputencoding=utf8,extendedchars=true,
        basicstyle=\tiny\bfseries,language={}]{#1}
    \end{tabular}
}

% the name of the programming language
\newcommand{\name}{\emph{TestScript}}
\newcommand{\pagelabel}[1]{\phantomsection\label{#1}}

% Set line spacing
\onehalfspacing


\begin{document}

\begin{titlepage}\begin{center}

    \vspace*{0.5cm}
    \Huge
    \textbf{The Development of a Programming Language}

    \vspace{1.5cm}

    % Include Mandelbrot on title page
    \ver{
    \begin{figure}[H]
        \centering
        \paste{paper/mandelbrot.tex}
        \caption{ASCII representation of the Mandelbrot Set, generated with \name{}}
        \label{f_mandelbrot}
    \end{figure}
    }


    \Large
    \vfill
    Matura Paper written by\\ \textbf{Florian M. Donnelly}
    
    \vspace{0.5cm}
    Under Supervision of\\ \textbf{Adrian Lüthi}

    \vspace{0.5cm}
    Gymnasium Burgdorf\\ October 31, 2022

\end{center}\end{titlepage}

\normalsize

\begin{abstract}
    This project is about developing a programming language and an accompanying
    interpreter. The goal is to learn about algorithms and techniques already
    popular in the industry while still creating something unique.
    The paper will explain the tools and procedures used to implement
    the different entities that make up the interpreter. 
    The interpreter is written in plain C in the gnu18 dialect, with
    the only library used being libc, the C standard library.

    The project including its structure and source will be presented and
    then be explained in more detail. A copy of the
    entire source code is contained in the appendix of this paper, starting on
    page \pageref{Appendix}, as well as on GitHub.com through the URL at the
    bottom of this page.

    \vfill

    By making use of just the term \emph{language} this paper will refer to programming
    languages specifically. Sources are mentioned after informationally coherent paragraphs in
    square brackets, an index can be found in the Bibliography on page
    \pageref{bibliography}.

    \vspace{2cm}

    \centerline{\begin{minipage}{\linewidth}

    The source code of the interpreter can also be found on GitHub: 

    \url{https://github.com/LeFlorians/TestScript}.

    The source code as of the latest commit before the submission of the paper
    is available at:

    \url{https://github.com/LeFlorians/TestScript/tree/\githash}.

    \end{minipage}}

\end{abstract}

{
    \ver{
        \small
        \newpage
        \hypersetup{hidelinks}
        \tableofcontents\newpage
    }
}

\normalsize
\onehalfspacing
\section{Introduction}
The importance of computers in today's world is unquestioned.
But computers are not just the metals and semiconductors they are built from,
but would be unusable without any software running on them.
Creating programs, no matter if games, web browsers or even operating systems,
involves the use of programming languages.
Those serve as a bridge between the human mind and its imagination, and
the capabilities of computer hardware, and are a huge time-saver when compared
to writing machine-native code, i.e. not using a programming language.

Since the year 1973, when the first programming language was invented, the
research in programming languages has vastly advanced. But the goal has stayed
the same, to make programming as intuitive and easy, yet fast and efficient as
possible.

A programming language is itself only a description or specification, i.e. a
set of grammar rules defining how single instructions and entire programs
can be written. Source code, which is a program written in a certain language,
has to be understood by the computer hardware in order to run. It can either
be translated to machine-native code with the help of a compiler, or
read line-by-line and interpreted by another program known as an interpreter.
Such compilers or interpreters are programs on their own that are classified
as programming language implementations. Subject of this paper will be to
create such a programming language implementation, or more specific, an interpreter
of the custom-designed \name{} language.

\subsection{Goals}
The focus of this project is learning about the internal design of interpreters
by creating a custom one that will be called the \name{} interpreter, building upon
strategies and algorithms used in already existent programming languages.

Modern programming languages often times provide additional tools aside from
the implementation. This includes dependency management tools to
easily create projects extending upon existing ones, or web platforms for
people to share their code, all of which are not subject of this paper.
The \name{} implementation will also not include many features seen in 
languages popular to date, as those are often created in years of time spent planning and
programming.

One more goal is to learn and understand the C programming language, in which
the interpreter of \name{} will be written in.

\section{Theory}
\subsection{Different Types of Interpreters}
In computing, there are several types of interpreters that behave differently.
Here are a few examples and their use-cases:
\begin{itemize}
    \item \emph{Just in time Compiler.} JIT-compilers blur the line between
        classical interpreters and compilers. As the name suggests, code is
        being compiled at run time of the program. The JIT-compiler can optimize
        compilation of more-often used code and therefore adapt itself to
        behave somewhat close to optimally for a given use-case. Today, this 
        technique is often used instead of classical interpreters, as
        JIT-compilers come with only advantages. The most popular example of
        a JIT-compiler would be Google's V8 JavaScript engine used
        in NodeJS and any Chromium browsers.
    \item \emph{Bytecode Interpreter.} Such interpret the bytecode that was
        output by a bytecode compiler from a given piece of code. The bytecode
        is an intermediate representation of the program, used to speed up
        both compilation and interpretation. Such an intermediary bytecode is
        useful for ensuring platform independence or portability, as it will be
        the same across any machine, the only difference being the interpreter.
        The most popular example would be the Java Virtual Machine,
        a bytecode interpreter for JVM bytecode, also used by many other languages
        such as Kotlin, Scala or Groovy.
    \item \emph{Abstract Syntax Tree interpreter.} Source code can be
        transformed into an AST or parse tree by a parser. A simple example
        can be found in section \ref{simple_interpreter}
        on page \pageref{simple_interpreter}. Compared to bytecode interpreters
        there is a large time overhead when performing syntax related computation
        or visiting tree nodes recursively.
    \item \emph{Self interpreter.} These are interpreters that interpret the
        language they themselves are written in. This requires a program
        written in a host language running an interpreter for the wanted
        language, in which runs another interpreter, for and written in the
        wanted language. Using a host language to initiate such a system is called
        bootstrapping.

        Self-interpreters are closely related to self-hosting compilers. These
        are compilers written in the language they compile.
        An example of a compiler written in itself is rustc, the Rust
        compiler, which was bootstrapped by the OCaml language.
\end{itemize}
[Source \cte{interpreters}]

This paper will implement a mixture form of an AST compiler and bytecode
interpreter. This has benefits, as an AST carries along lots of information
that is not needed in later stages of interpretation and can be discarded.
The AST is converted to an intermediary bytecode format which only contains
necessary information about the program. Such bytecode can then be cached,
e.g. kept in memory, meaning it only has to be converted once and can be
used many times, also leading to an improvement in performance.
[Sources \cte{V8}, \cte{jvm}, \cte{rustc}]

\section{Implementation}
\subsection{Used Software}
At the heart of making any project in the C programming language
stands the C compiler. This project uses
\emph{gcc}, the GNU Compiler Collection, which, as the name suggests, also 
supports more languages than solely C. Substantiating the development process is 
\emph{GNU Make}, a general-purpose build system. It simplifies repetitive tasks,
removing the need to write and remember long build commands. Building a project
means to convert its source code into an executable program.
The ecosystem of C language related tools is fundamentally supported by 
\emph{GNU+Linux}, the operating system used to program on.
The code editor used was \emph{Microsoft Visual Studio Code}.
It has built-in git integration and auto-completion features.
The tools \emph{Valgrind} and \emph{gdb}, the GNU Project Debugger, were of help
when trying to find errors and memory leaks, see section \ref{memleaks} for details.
\emph{Git} is the project management and version tracking software used to back up and share
the project files on \emph{GitHub.com}, a Microsoft hosted git repository and
web front-end to store and collaborate on source code.

\subsection{Simplyfing the Problem}
When dealing with problems in general, it is helpful to map out and think about
the way of approaching the problem first. A common approach is to distill the
problem situation down to a simpler one and extending the solution to take into
account all important aspects later.

The problem to solve in this paper is concisely to make a computer understand
human written and -readable text, called source code, and follow the instructions
it proposes.
Most interpreters follow a plain scheme to go about this problem, but the
actual implementation of each of the steps can be heavily customized.
Here is a possible approach on implementing an interpreter, broken
down into three steps:
\begin{enumerate}\pagelabel{simple_interpreter}
    \item To create meaningful groups of characters from the input string. This
        process is known as tokenization and is done with the component called
        tokenizer or lexer. As an example, the string \emph{ab + cd} could result
        in three tokens, namely \emph{ab}, \emph{+} and \emph{cd}. In this case, space characters
        are simply ignored and could be left out, unlike in the English language where spaces are relevant. 
        This is only an example and the grouping of characters could be done 
        in any thinkable, logical way.
    \item To order the tokens into a parse-tree, also called syntax-tree, as
        specified by rules. These rules are called a grammar.
        Creating an order is important, as some calculations depend on others, e.g. multiplication
        is always done before addition. As the name suggests, a data structure
        called tree, often times a binary tree, is used.
    \item To interpret the parse tree, traversing through it and following the
        instructions each of the tree-nodes represent. The outcome
        will be the result of the computation the program describes.
\end{enumerate}

As seen above, these components, or stages as they will be called in this paper,
each input and output data, where the output of one stage is the input of the
respective next stage.

\newpage\subsection{A Project Overview}
In the following paragraphs, the project in its entirety will be briefly
presented, giving an overview over files and correlations.

\subsubsection{Flowchart}

\ver{
\begin{figure}[H]
    \centering
    \includegraphics[width=0.8\textwidth]{paper/Data Flow Diagram.png}
    \caption{A \name{} Data Flow Diagram}
    \label{f_diagram}
\end{figure}
}

The above figure is a representation of how different components of the entire
\name{} implementation interact and work together. Each next stage performs
operations on the output data of the proper. The \hrc{src/interpreter/interpreter.c}
file, here depicted as 'Interpreter', chains all stages together.


\subsubsection{File Structure}\label{FileStructure}
\begin{minipage}{\textwidth}

Here is a file tree overview of the project which will be referenced
several times in the following sections of this paper. In the depicted file tree
only source- or configuration files are shown. No output files, executable files or files that
are of no further importance regarding \name{} are depicted.
\ver{\begin{flushleft}
    \scriptsize\setlength{\DTbaselineskip}{7pt}
    \dirtree{%
        .1 project root.
        .2 \hrc{.gitignore}.
        .2 \hrc{README.md}.
        .2 \hr{Makefile}{Makefile}.
        .2 \hrc{mapop.gperf}.
        .2 pseudocode.
            .3 \hrc{pseudocode/pratt-parser.c}.
            .3 \hrc{pseudocode/function.c}.
        .2 examples.
            .3 \hrc{examples/mandelbrot.nts}.
            .3 \hrc{examples/montecarlo.nts}.
            .3 \hrc{examples/fibonacci.nts}.
            .3 \hrc{examples/exec.nts}.
        .2 src.
            .3 \hrc{src/main.c}.
            .3 interpreter.
                .4 \hrc{src/interpreter/bytecode.c}.
                .4 \hrc{src/interpreter/bytecode.h}.
                .4 error.
                    .5 \hrc{src/interpreter/error/error.c}.
                    .5 \hrc{src/interpreter/error/error.h}.
                .4 \hrc{src/interpreter/interpreter.c}.
                .4 \hrc{src/interpreter/interpreter.h}.
                .4 libraries.
                    .5 \hrc{src/interpreter/libraries/libraries.c}.
                    .5 \hrc{src/interpreter/libraries/libraries.h}.
                    .5 stdlib.
                        .6 \hrc{src/interpreter/libraries/stdlib/stdlib.c}.
                        .6 \hrc{src/interpreter/libraries/stdlib/stdlib.h}.
                .4 \hrc{src/interpreter/localizer.c}.
                .4 \hrc{src/interpreter/localizer.h}.
                .4 mappings.
                    .5 \hrc{src/interpreter/mappings/operations.c}.
                    .5 \hrc{src/interpreter/mappings/operations.h}.
                    .5 \hrc{src/interpreter/mappings/mapop.c}.
                    .5 \hrc{src/interpreter/mappings/mapop.h}.
                .4 memory.
                    .5 \hrc{src/interpreter/memory/hashtable.c}.
                    .5 \hrc{src/interpreter/memory/hashtable.h}.
                    .5 \hrc{src/interpreter/memory/array.c}.
                    .5 \hrc{src/interpreter/memory/array.h}.
                    .5 \hrc{src/interpreter/memory/stack.c}.
                    .5 \hrc{src/interpreter/memory/stack.h}.
                .4 \hrc{src/interpreter/parser.c}.
                .4 \hrc{src/interpreter/parser.h}.
                .4 processing.
                    .5 \hrc{src/interpreter/processing/implementations.c}.
                    .5 \hrc{src/interpreter/processing/implementations.h}.
                .4 \hrc{src/interpreter/runtime.c}.
                .4 \hrc{src/interpreter/runtime.h}.
                .4 \hrc{src/interpreter/tokenizer.c}.
                .4 \hrc{src/interpreter/tokenizer.h}.
    }
\end{flushleft}}

\end{minipage}
\linebreak

As visible in the file tree, most of the source-code files, ending in .c,
have a corresponding header file ending in .h. This is due to the nature
of the compilation process of the C language, which is the host language
the \name{} interpreter is written in.
Each header file contains only information about functions, types or variables
available to other source or header files. The actual implementations or 
variable values are stored in the respective .c files.

The file named \hr{Makefile}{Makefile} carries information for the program \emph{GNU make}.
It contains built targets, which are specifications on how to compile
certain parts of the project. When combined, these targets can build the
entire project, taking the source code and compiling it to machine-native code.
These build targets contain shell commands, which would otherwise have to be
typed out in their entirety every time the project should be built. 
[Source \cte{makefile_tutorial}]

A short guide on how to build the \name{} interpreter can be found in 
the file \hrc{README.md}. The file name is spelled in capital letters by convention.
It tells users about
which programs have to be preinstalled to compile the project and which \emph{GNU make}
commands are used to build or clean up the project.
It also contains some information about all the available standard library
functions that are included. The standard library is a set of functions
that are part of the interpreter and that extend its functionality. 

The file \hrc{mapop.gperf} is also a source file. It is used in conjunction with the \emph{GNU gperf}
tool. The tool creates a perfect hash function, which is output to the file
\hrc{src/interpreter/mappings/mapop.c}. The perfect hash function is used to translate
single- or multi-character symbols, such as \emph{!=} and \emph{+}, to their respective
interpreter internal representation in constant time.
[Sources \cte{c_compilation}, \cte{gperf}, \cte{perfect_hash}, \cte{hash_functions}]

\subsection{Implementing \name{}}
This section will explain the process of constructing, and the inner workings of
the \name{} interpreter. It is therefore the most complicated and detailed section
in this paper. Examples will be made with both pseudocode, i.e. code for
illustration purposes, and actual, working source
code from the project to show the decisions made during
interpreter design. Only the file names are used, as those are unique. To view the
entire file paths, look up the file names in the file tree on page \pageref{FileStructure}.

The syntax of \name{} is not specified in a formal programming language grammar,
but rather implemented from scratch in the files \hrc{src/interpreter/tokenizer.c}
and \hrc{src/interpreter/parser.c}.

\centerline{\begin{minipage}[H]{\linewidth}
\subsubsection{The Entry-Point}
The entry point of the project lies in the \hrc{src/main.c} file.
The file implements the C main function, which is the entry-point of any
C application. Since the file is so small, here is its source code:
\code{src/main.c}{1}{999}{C}

\end{minipage}}

For the readers who do not know how to read source code, C in particular, here is a short
explanation of the basic syntax of the C language taking the \hrc{src/main.c} file
as an example:
\begin{itemize}
    \item \emph{Preprocessor Instructions.} Lines beginning with hash symbols (\#) are
        C preprocessor instructions and are dealt with by the C preprocessor.
        Preprocessing is what the compiler does before actually compiling the
        code.

        The most common preprocessor instructions are the following two.
        On one hand, there is \expr{\#include $<$header.h$>$}
        to include various definitions from a header file, i.e. to paste
        its content at where the \emph{include} statement lies.
        On the other hand exists \expr{\#define A B} used to
        define \emph{A} as an alias for \emph{B} and replace every following occurrence of \emph{A}
        with \emph{B}.

    \item \emph{Functions.} In programming, functions, or methods, are
        a set of code that accomplish a certain task. They take in data,
        process it and return a result. Calling a function means to evaluate it for a given input.
        In C, every function has a return type, a
        function name, arguments and a function body in that order.

        \centerline{\begin{minipage}[H]{\linewidth}
        Here is an example of a C function:
        \code{pseudocode/function.c}{1}{999}{C}
        \end{minipage}}
        
        This is a main function, similar to the one
        in \hrc{src/main.c}. It returns a value
        of type \emph{int}, has two arguments, \emph{argc} and \emph{argv}, and a function body in
        which the value of the \emph{argc} argument is printed out.

    \item \emph{Pointers.} Pointers are variables that store memory addresses. 
        In C syntax they can be defined with the star (*) symbol.
        An example from \hrc{src/main.c} is \expr{FILE *input;.} That means that \emph{input} is not
        actually a FILE, but rather a memory address pointing to a spot in memory where a FILE is.
        To dereference the pointer, i.e. to copy the value stored
        at the memory address into another variable, one more star is used like this:
        \expr{FILE value = *input;.}
        \emph{value} now contains a copy of the actual FILE that \emph{input} was pointing to.
\end{itemize}
[Source \cte{cpreprocessor}]

For those who are further interested in the C programming language, there are
countless internet articles or tutorials about reading and understanding, or
writing and compiling C code. One such example is the article \emph{The C Beginner's
Handbook} provided by freeCodeCamp.org linked in the sources
on page \pageref{bibliography}.
[Source \cte{freeCodeCamp}]

The next few sentences are a summary of all
things happening in the \hrc{src/main.c} file.
First, the \hrc{src/interpreter/interpreter.h} header file is included. 
The main function then decides whether to
read a program from a file or from standard input. 
At last, the function \emph{interpret}
from the included header is called with the selected input and
interpretation can begin.

\subsubsection{The Interpreter}
This section deals with the file 
\hrc{src/interpreter/interpreter.c} which guides
input data through the various stages of the \name{} interpreter as 
seen in figure \ref{f_diagram} on page \pageref{f_diagram}.
Following is a breakdown of the \emph{interpret} function from said file
as well as an overview of some of the sections that follow.

The function \emph{interpret} calls the parser to generate a syntax-tree from tokens retrieved
from the tokenizer, implemented in \hrc{src/interpreter/parser.c} and 
\hrc{src/interpreter/tokenizer.c} respectively.

After that, the names of variables are
mapped to addresses in memory where their value is stored. 
Doing this before
bytecode compilation has a performance benefit as the memory addresses can be
cached and reused. 
The file walking the AST and replacing
each variable name with a memory address is called \hrc{src/interpreter/localizer.c}. The name
localizer is not an industry standard and was invented for this project. 

The mapping is achieved with a custom hash table data structure with average time
complexity of O(log n), n being the total amount of variables a \name{} program is using.
The hash table implementation used in \hrc{src/interpreter/memory/hashtable.c}
uses a binary search algorithm to locate elements in variable-sized arrays in the hash table's buckets.
This is because for such few variables as are used in a standard \name{} program,
an interpolation search algorithm actually performs worse.
For collision detection, i.e. to circumvent problems arising when two variable names have the
same hash, a technique called separate chaining is used. In this approach, linked lists with
key-value pairs are used for each array index of the buckets and variable names are compared directly
to find exact matches.
[Source \cte{separate_chaining}]

The syntax tree from the parser could now be interpreted directly, but as mentioned 
before, is instead converted
to an intermediate internal bytecode format by \hrc{src/interpreter/bytecode.c}.
This step is done so that code that is run multiple times does not cause 
a large time overhead when traversing the syntax tree.
The final step is to process or execute said bytecode. This step is performed
in \hrc{src/interpreter/runtime.c}.

\subsubsection{The Tokenizer}
The tokenizer is implemented in \hrc{src/interpreter/tokenizer.c}. Its job is to
take source code in plain-text format as input and output so-called tokens.
These tokens represent groups of characters that belong together. In this paper's
implementation, each token has a strict type and a text content. In an IT context,
a text or group of characters is often reffered to as string.
The different types of tokens are defined in the header file
\hrc{src/interpreter/tokenizer.h}.

Following is an example of tokenization.
An input of \emph{a.b==23.4} would result in the the tokens depicted below according
to the tokenizer implementation of this paper:
\begin{enumerate}
    \item A token of type FIELD with content \emph{a.b}.
    \item A second token of type SYMBOL with content \emph{==}.
    \item A third token of type NUMBER with content \emph{23.4}.
\end{enumerate}
FIELD type tokens are later replaced by REFERENCE tokens, effectively mapping
variable names to addresses in memory.
SYMBOL type tokens are mapped to operators such as plus or minus by \hrc{src/interpreter/mappings/mapop.c}
during parsing. NUMBER type tokens are replaced by an actual number, for which
C's \emph{long double} data type is used as specified in \hrc{src/interpreter/bytecode.h}.

While tokenizers can be programmed using regular expressions, 
this paper uses a simpler \emph{deterministic finite-state machine tokenizer}.
It reads character by character, determining the token type as it goes.
It stops when there is a character that does no longer conform to the rules of the current token type.
Tokens are each requested and integrated into a syntax-tree by the parser as described in the next
section. 
[Source \cte{dfsm}]

\subsubsection{The Parser}
The parser, implemented in \hrc{src/interpreter/parser.c}, generates a syntax tree,
also called parse tree, from tokens. Since this paper's implementation 
does not parse code given a
formal context free grammar, the syntax tree the parser creates is very
similar to an otherwise known abstract syntax tree, AST for short.
ASTs describe source code conceptually and do not contain all syntactical
elements required to parse code.
In this paper the terms parse tree, syntax tree and AST will be used
synonymously.

The algorithm used is inspired by \emph{the Pratt Parsing
algorithm} first described by Vaughan Pratt in 1973, which is a kind of precedence
parser based on recursive descent. There is an explanation of the algorithm
written by Jonathan Apodaca on \emph{dev.io} linked in the references on page 
\pageref{bibliography} which this paper's parser takes inspiration from.
[Sources \cte{pratt}, \cte{devio}, \cte{parsing_guide}]

The syntax tree in the implementation of \name{} is a binary tree consisting of nodes, internally
called \emph{stnode}. The datatype is defined in the header file \hrc{src/interpreter/parser.h}.
Each node can either be an internal node having one or two nodes as children, or a leaf node
which stands for a value, e.g. a number or a text. Each internal node, also known as parent node, signifies
an operation done on its children. This can be for example the addition of two numbers. 

The difficult part of implementing a parser is precedence. Precedence together
with associativity define the order of execution of statements. A simple example
would be to perform multiplication before addition, e.g. \emph{2+3*4} would
result in \emph{14} and is not the same as \emph{(2+3)*4}.
Associativity is either \emph{Left to Right} or \emph{Right to Left}. An example of associativity is
\emph{20/10/2}. Both in C and in \name{}, the division operator is
left-to-right associative, meaning the stated expression is the same as
\emph{(20/10)/2} and not as \emph{20/(10/2)}. Both precedence 
and associativity are properties that are defined in the file \hrc{mapop.gperf} for
each operator.
In the end, the parse tree contains all values and 
operations as well as information about the order of execution.

The parser implementation in \hrc{src/interpreter/parser.c} may be difficult to
read and understand, which is why the most important functions are explained here:
\begin{itemize}
    \item \emph{advance.} This function reads the next token from the tokenizer.
    \item \emph{secondary.} This function reads tokens that will become the leaf nodes
        of the parse tree.
        It deals with simple values that make up the start of an expression, such as numbers,
        fields, strings, and it handles brackets. 
        In addition, it can handle prefix operators, such as the unary plus or the unary minus.
    \item \emph{expr.} The \emph{expr} function calls \emph{secondary}, the result of which
        acts as the basis for a new expression. It then repeatedly checks
        if an operator follows the secondary token and creates the syntax-tree
        with help of recursive mechanisms. The nature of this function also 
        removes the need for semicolons,
        which are essential in languages like Java or C.
    \item \emph{parse.} The parse function is a small wrapper which only
        calls the \emph{expr} function a first time, initiating the parsing process.
\end{itemize}

\centerline{\begin{minipage}{\linewidth}
Following is a piece of pseudocode to better illustrate the inner workings of the
parser which is based on a \emph{Pratt Parser}.
This is only an example and not completely accurate to the actual parser used
in this project.

\code{pseudocode/pratt-parser.c}{1}{999}{C}
\end{minipage}}

\subsubsection{The Bytecode Compiler}
The bytecode compiler is an internal part of the interpreter and is located in the file
\hrc{src/interpreter/bytecode.c}. Its job is to create a list
of byte-sized instructions and data from the syntax-tree the parser created.
Using such an intermediary bytecode format instead of directly interpreting
the parse-tree can be beneficial. This is the case when the same piece of code
has to be run several times. Compared to
having to walk along tree nodes recursively, reading the list of instructions in
the bytecode takes less time overall. 

Not to be confused with machine-native code, the bytecode compiler in \name{}
generates byte-sequences in an internal format that are then interpreted. A
Just-In-Time compiler, JIT compiler for short, would instead compile the AST 
down to machine-native code, which
runs directly on the system's hardware and does not have to be interpreted.

To generate the bytecode, \name{} recursively reads through the
binary tree created by the parser and pushes found instructions onto a stack.
The algorithm used is inspired by Edsger Dijkstra's \emph{Shunting yard 
algorithm}, traversing the tree post-order to create
a prefix notation. 
A prefix notation is a way to write down mathematical expressions, except
operators precede their operands in contrast to the more common infix notation,
where operands surround the operator.

In \hrc{src/interpreter/bytecode.c} we can see that first the left operand and then
the right operand are added onto the stack of bytecode. Therefore, when
popping from the stack and evaluating instructions, the operands will have to
be read in reverse.
Here is an example:
\begin{enumerate}
    \item Assume the following mathematical expression in infix notation:
        \expr{\emph{1 - 2 * 4 / (5 + 6)}}
    \item In common prefix notation, also called Polish notation, the
        operators would now precede the operands in this manner:
        \expr{\emph{- 1 (/ (* 2 4) (+ 5 6))}}
    \item But in \name{} the operands are pushed onto the stack is reverse order,
        as popping from the stack again reverses the order back to the original.
        This would look like the following:
        \expr{\emph{- (/ (+ 6 5) (* 4 2)) 1}}
    \item The brackets are not needed, because for every operator the amount
        of operands is known. This is also the case for the unary minus and unary plus,
        because they are distinguished from their infix counterparts beforehand.
        The above statement would therefore look like this:
        \expr{\emph{- / + 6 5 * 4 2 1}}
        This is how expressions are represented in the bytecode of \name{}.
\end{enumerate}
[Sources \cte{polish_notation}, \cte{shunting_yard}]

\subsubsection{Calculating Results}
This is the last stage of interpreting program code as implemented in \name{}.
The interpreter steps through the intermediary bytecode created by the bytecode
compiler and evaluates expressions by plugging in the provided values.
All available operators are implemented in 
\hrc{src/interpreter/processing/implementations.c}.

The evaluation of expressions works in a recursive manner. Here is an 
example of \name{}'s bytecode notation and how it is evaluated:
\begin{itemize}
    \item Assume the following input in \name{}'s bytecode format: 
        \expr{\emph{- 6 * 3 4}}
        In common infix notation, this would be equivalent to 
        \expr{\emph{(4 * 3) - 6}.}
        The question arises, how can an algorithm compute the result of this
        calculation? The answer is recursion.
    \item The interpreter will read through the bytecode from left to right. First, it will
        notice the subtraction operator and accordingly assume that two
        operands must follow.
    \item The function then recursively calls itself in order to evaluate the
        right operand and notice the 6, a value and a leaf node in the parse tree. 
        The function simply returns the 6 to the subtraction expression.
    \item Again, the function calls itself to find the next operand.
        As it now reaches a multiplication operator,
        it recursively reads through that sub-expression's right and then left
        operator, which are 3 and 4 respectively.
        The result is computed and returned.
    \item Now the subtraction knows its right- and left-hand operators, 6 and 12.
        The subtraction is performed and the result, 6, is returned.
\end{itemize}

That is it. With all the operators and standard functions that have been implemented, entire programs
can be interpreted. Examples of such programs can be found in the directory
\emph{examples} of the source code.

\section{Results}

\name{} is built upon the implementation of a recursive Pratt Parser
and a basic tokenizer. It uses a hash table to map variable names to their
memory addresses. There are no keywords, only a set of non-alphanumeric characters
representing all possible operations. Extending upon that, there are some
standard functions that are included in the interpreter and that enlarge the capabilities of the language.
The following subsections display examples of the syntax and of the language and
how programs are written in it.

\subsection{The Fibonacci Sequence}
The syntax of \name{} is unlike most modern, common programming languages' syntax,
but comes closest to that of JavaScript.
In \name{}, there are no keywords at all and everything is controlled via operators,
which are non-alphanumeric characters. Such operators include the plus-sign 
for addition, the equals-sign for variable assignment or a set of brackets to signal
a function call.

Following is a piece of example \name{} code, printing out the first ten numbers
of the Fibonacci sequence, as well as a dissection of how that same code is interpreted.
\code{examples/fibonacci.nts}{1}{999}{tslang}
\begin{itemize}
\item The source code begins with the definition two variables \emph{a} and \emph{b}, which
    will be utilized to keep track of the data used to calculate Fibonacci numbers.
    They are each assigned a number, zero and one respectively.
    Since the variable names \emph{a} and \emph{b} are not yet
    known to the interpreter, memory is automatically allocated for them
    and initiated to a default value. That default value is then overwritten with the
    assignment operator.
\item Another variable \emph{count} is declared and defined as the number ten on line 4.
    This variable will be used to limit the amount of numbers the program outputs.
\item Now the function \emph{std.repeat} is called. This function is a
    standard library function, meaning it is included in the \name{} interpreter and
    written in C. A so-called \emph{anonymous function} is passed as argument to the function call.
    It is called anonymous because it is not bound to an identifier. This anonymous function
    will be repeated by \emph{std.repeat}.
\item On line 7 and 8, some calculations are done.
    Remember, the equals sign stands for an assignment, not an
    equation. First, \emph{a} is assigned the sum of \emph{a} and \emph{b}. Then 
    \emph{b} is set to be the new \emph{a} minus \emph{b}, 
    which is again the same as \emph{a} before the reassignment on line 7.
\item On lines 10 and 11, the standard library function \emph{std.print} is used
    to print out the value of a and a line break. The backslash in the text
    that is print out signifies a so-called escape sequence and will be replaced with
    another piece of text that can not be entered directly. The \emph{n} stands
    for a line break, but other letters such as a \emph{t} for a tabulator are possible.
\item Lastly, line 13 contains another calculation in which the variable \emph{count}
    is decremented by one. Since this is the last instruction of the anonymous function,
    it simultaneously is its return value, e.g. the value of count is returned
    to the \emph{std.repeat} function. The \emph{std.repeat} function will
    repeat the inner function until the returned value is zero.
    This means the inner function is executed exactly ten times, because
    the initial value of count has to be decremented by one ten times until
    it equals zero.
\end{itemize}
[Source \cte{anonymous}]

\subsection{Name Shadowing}
Here is another example, which in contrast to the example above, focuses on an
original concept implemented in \name{}.
\code{examples/exec.nts}{1}{999}{tslang}
The code begins by creating the variable \emph{f}.
The function it holds is a wrapper around the standard library function \emph{std.exec}.
It executes a command in the user's shell. A shell is a program that
lets users interact with the operating system, most often through a text-based interface.
Common shells include \emph{cmd.exe}
on Microsoft Windows or \emph{GNU Bash} on most Linux distributions.
The command that will be executed in the shell is \emph{'echo Hello There!'},
which will simply output the text 'Hello There!'.
As the comment on line 5 explains, the function call on line 6 of the example
program opens a YouTube video in Firefox. Because the start command is not
available in \emph{GNU Bash}, this instruction will most likely not work
on GNU+Linux.
Because \emph{std.exec} could be used for malicious purposes, such as
shutting down the computer or deleting important files, access to that
function can be revoked via a hash symbol as on line 12. The instruction
on line 15 that is supposed to turn off the machine will therefore not work
anymore. 
The interesting thing is that the function call to \emph{f} in line 18
will still work. This is because a hash symbol does not simply delete
a variable, but rather make it inaccessible for future references. To sum up,
the function \emph{f} is still capable of using \emph{std.exec}, because
at the time \emph{f} was defined, \emph{std.exec} was still available.

This security concept forces developers to think about access restriction. First, safe to use aliases
to dangerous functions have to be created. Next, access to all exploitable functions is
revoked. This way, any code can only access variables and functions that are still available.

\section{Discussion}
The name \name{} was chosen, because it implies this programming language
to be a learning project as well as a scripting language, which employ
high-level abstractions of underlying operations and interpret one line
at a time. The file name extension \emph{.nts} is used in the examples. This
file extension is not mandatory, but was chosen as it stands for \emph{Not TypeScript}, because
the language neither enforces strict typing, nor is it Microsoft's \emph{TypeScript}.

The following paragraphs will go over some positive results and some caveats of the
final product.

\subsection{Limitations}\label{Limitations}
Unfortunately, there are limitations that make \name{} practically unusable for actual
professional projects. In this section, a few shortcomings of \name{} are listed
and discussed.

\subsubsection{Numbers}
Another goal that turned out not to be practical to achieve was the use of just as much
memory for each variable as needed and not more. This means every variable would
have to be assigned a datatype, containing information about the size and domain of possible values.
Taking C as a reference, the datatype \emph{char} is exactly one byte in size, and
can therefore be in exactly $2^8$ or 256 different states. But in C, information about
the datatype can be discarded at compile time, whereas in an interpreter like \name{} such
information needs to persist throughout the program's lifetime. This is because interpreters
do not know the entire source code of the program before it is executed, while compilers do.

\subsubsection{Memory Leaks}\label{memleaks}
Memory leaks happen when a program reserves memory to store data, but
never frees up the space to be used by other programs, even though the data
has long been forgotten about and will no longer be accessed.
In \name{}, memory leaks are a major problem. Even though they are barely noticeable in small
programs it interprets, \name{} does not have proper setups in place to free up
all used memory that is not used any longer.

Many programming languages, such as Java and Python,
make use of Garbage-Collection, a strategy to track unused allocated memory and
free it up automatically. While \name{} also tries to make use of this technique,
there are many flaws that were introduced while developing the interpreter that hinder 
Garbage-Collection from working correctly.

One reason for memory leaks is the nature of the
C programming language, in which the \name{} interpreter was written in. The C language and
compiler do not keep track of all possible states in which certain blocks of memory are
forgotten about and do not warn the developer. Instead, memory leaks happen and a program
uses much more system resources than required.
Modern alternatives for C such as Rust deal with the problem in a more elegant way. 
Rust has an ownership system in place,
in which any allocated memory is owned by one function and can be borrowed by 
multiple other functions. When the owning function
is done processing, the memory is automatically freed. Such a system
is a helpful tool for developers to create better programs without having to get
lost in the details.
Coming up with new ways to make the job of developers easier is what programming
language design is all about.

A tool that is helpful to find and fix memory leaks is \emph{Valgrind}. It runs
the executable program against an input and finds the origins of not freed data.
Even though it was used to fix some memory leaks, many remain.

\subsubsection{Single-Threaded}
The \name{} interpreter does not implement any multi-threading or multiprocessing.
These techniques make use of the multiple CPU cores available in modern computers
to run programs in parallel.
Any \name{} program can therefore
only contain instructions that run sequentially.
Since most languages used today allow increasing a
program's throughput using multi-threading, not having the feature at hand in \name{}
is a major limitation.

\subsubsection{Lack of a Development Environment}
This is probably the largest problem of \name{} and new programming
languages in general.

Most widely used programming languages not only consist
of a language specification and a compiler or interpreter, but of many more
tools that surround them.
Such tools can include a package-manager, with which people around the world
can share code. Examples would be \emph{npm} or \emph{cargo}, 
the \emph{Node Package Manager} for JavaScript applications and Rust's package 
manager respectively.
Another example of a helpful tool is a debugger, helping developers
find the cause of errors, so-called bugs, in their programs.
This includes the tool gdb, the GNU Project Debugger, that was used in this
project to find errors in the \name{} implementation.

On the other hand, having to deal with countless package-managers for each
language can be annoying for developers. A solution is to build upon and
extend already existing languages and their tools with for example a new syntax.
This is exactly what Kotlin does. Kotlin can produce Java Virtual Machine
compatible bytecode, can be transcompiled, i.e. translated, into
JavaScript, or ECMAScript 5.1 to be more specific,
and it can be compiled to machine-native programs via LLVM. 
[Sources \cte{npm}, \cte{cargo}, \cte{kotlin_compile}, \cte{llvm}]

\subsection{Interesting Ideas untouched}
Creating the ideal programming language takes a lot of planning and sketching
out. This section of the paper gives insight on interesting ideas or concepts
that are used in other programming languages, but 
could not be implemented in \name{}.

\subsubsection{Foreign Function Interfaces}
A foreign function interface, often abbreviated as FFI,
is a mechanism by which one programming language can call functions written 
in another. This makes a language more attractive to new users, as it can be used
in conjunction with existing libraries from other languages. 
Enabling calls to foreign C function could allow programms written in a new langugage
to directly interact with system libraries or the operating system itself.

Foreign function interfaces are available in many modern languages.
Examples are Java, Python and Rust.
Java enables users to call C, C++ and even assembly code with the Java Native
Interface, JNI for short, whereas within Python C function can
be run with the standard library \emph{ctypes}.

Foreign function interfaces are very difficult to implement correctly.
Let us assume a programming language interpreter \name{} written in C that
would like to call some functions that are part of yet another C program or
library. The problem lies in the fact that at runtime, C does not know the
structure of the functions to be called, say return- or argument-types, which
is by the way where the term \emph{foreign functions} originates from. This is why the
\name{} developer would have to provide that information by re-declaring the
function inside of \name{}, so that, with some trickery, data can be passed to and
received from the function in the correct format, preventing a segmentation
fault. The libffi C-library, which is also used by Java and
Python for this purpose, can be used to load and call such foreign functions.
The library was not used in this paper's implementation, as the compilation process
on Microsoft Windows would have become too difficult.
[Sources \cte{FFI}, \cte{JNI}, \cte{ctypes}, \cte{libffi}]

\subsubsection{Reflection}
Reflection, also called reflective programming, describes
the ability of a program to examine and modify its own structure and behavior 
at runtime. An example would be self-modifying code. This can easily be
achieved in assembly language, which inherently does not differentiate between
data and instructions.
Another example is Java, providing methods and classes in the java.lang.reflect
package that enable developers to examine and change properties or functions
of objects.
The easiest and simplest way to go about implementing reflection in \name{} would 
certainly be to allow the interpretation of strings, i.e. text, at runtime. That way, strings
holding program instructions can be created and modified and later executed.
The difficulty of implementing reflection in the current state of \name{} 
lies in the fact that either syntax trees that exist already would have to be 
altered according to the injected source code, or syntactical compromises would have to be made.
This is why reflection is not a feature of \name{}.
[Source \cte{reflection}]

\subsection{Conclusion}
Creating an interpreter was a challenge. From the beginning on, the project
had to be planned and commented strictly, not to get lost in the complex web
of source code later on. The final product is a success, a working interpreter
hosting a custom programming language that can run human-readable programs.
Most parts of the source code are annotated with comments, giving interested
readers direct explanations of what is going on.
Although there are several limitations and shortcomings as stated in section \ref{Limitations}
of this paper, the details of the implementation brought about unexpected 
problems and solutions. The goal of this project to learn about 
the inner workings of an interpreter has been achieved.
With the source code of the project having exceeded 3500 lines, getting to
know the C programming language better was a success as well.

\section{Index}
This section contains a bibliography which links to all online articles
this project could profit from and the respective dates of last access, and the list of figures.
The entire source code of \name{} can be found in the Appendix, section \ref{Appendix}.

\subsection{List of Figures}
{
    \hypersetup{hidelinks}
    \listoffigures
}
\vspace{1cm}
Figure \ref{f_mandelbrot} was generated using a \name{} program, the source code
can be found in the file \hrc{examples/mandelbrot.nts}.
Figure \ref{f_diagram} was created using an online diagram creation tool available
on \url{https://app.diagrams.net}.
[Source \cte{drawio}]

\subsection{Bibliography}
\renewcommand*{\bibfont}{\normalsize}
\pagelabel{bibliography}\printbibliography[heading=none]

\section{Declaration of Authenticity}
Hereby I, Florian M. Donnelly, declare to have written this paper independently, 
also including the entire source code of \name{}, using resources 
listed under the \emph{Bibliography} section on page \pageref{bibliography}.

\section{Appendix}\label{Appendix}
The appendix contains the complete source code of \name{}.
See the file tree under \emph{File Structure} in section \ref{FileStructure}
to get an overview. In the digital PDF version of this paper, all file names
in the text and the file tree are clickable and will open the file in the appendix.

% Append all soruce files
\ver{
    \foreach \fname in \files{\sourcecode{\fname}{1}{9999}{C}}
    \sourcecode{Makefile}{1}{9999}{make}
    \sourcecode{README.md}{1}{9999}{TeX}
    \sourcecode{.gitignore}{1}{9999}{tslang}
    \sourcecode{mapop.gperf}{1}{9999}{tslang}
    \sourcecode{examples/mandelbrot.nts}{1}{9999}{tslang}
    \sourcecode{examples/montecarlo.nts}{1}{9999}{tslang}
    \sourcecode{examples/fibonacci.nts}{1}{9999}{tslang}
    \sourcecode{examples/exec.nts}{1}{9999}{tslang}
}

\end{document}

